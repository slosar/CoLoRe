\documentclass[a4paper,10pt]{article}
\usepackage[utf8]{inputenc}
\usepackage{amsmath}
\usepackage{fullpage}

%opening
\title{Fast lognormal realizations for multi-probe experiments}
\author{David Alonso}

\begin{document}

\maketitle

\section{Notation}
$\chi$ is comoving radial density. We use units $c=1$ throughout.

\section{Algorithm}
  \begin{itemize}
    \item Generate a Gaussian realization of the linear density field
      $\delta_G$ and the newtonian potential $\phi$ at $z=0$. For
      this we use FFTW in a box able to hold a sufficiently large
      sphere. In what follows, let $dx$ be the comoving resolution 
      used for this.
    \item Interpolate cartesian grid into spherical shells. These
      are generated with a width $dr=dx$. The pixels are defined
      using a CEA scheme (cylindrical equal area), with constant
      intervals in $\phi$ and $\mu\equiv \cos\theta$.

      Let $N_\mu$ be the number of divisions used in $\mu$. Firstly,
      we will use $2\times N_\mu$ divisions in $\phi$. Secondly,
      the largest transverse scale covered by each pixel is given by
      \begin{equation}
        \Delta d_{\rm max} = \chi\,{\rm arccos}(1-2/N_\mu),
      \end{equation}
      where $\chi$ is the comoving distance to the shell.
      We determine $N_\mu$ by trying to achieve $\Delta d_{\rm max}=dx$,
      which implies $N_\mu={\rm ceiling}(2/[1-\cos(dx/\chi)])$.
    \item While doing the interpolation we compute the relevant
      quantities at $z=0$ that will be used later. These are the radial
      velocity, $v_r$, proportional to the radial gradient of $\phi$, and
      the transverse tidal field, given by the angular second derivatives
      of $\phi$. These two objects are needed for the computation of RSDs
      and lensing. We compute these quantities by differentiating through
      finite differences in the Cartesian grid, rotating and interpolating
      into the spherical shells (in this order).
    \item These fields are then evolved in the lightcone in different
      ways (see details below). Galaxies are Poisson-sampled according
      to the evolved, bias and lognormalized value of the density 
      field, and are assigned redshift and shape distortions according
      to the evolved velocity and lightcone-integrated transverse tidal
      field.
  \end{itemize}

\section{Relevant equations}
  \begin{align}
    &\phi_{\bf k}(z=0)=-\frac{3}{2}\Omega_MH_0^2
    \frac{\delta_{\bf k}(z=0)}{k^2},\hspace{12pt}
    v_r(z=0)=-\frac{2\,f_0}{3\,H_0\Omega_M}\hat{\bf n}\cdot\nabla\phi(z=0)\\
    &\delta(z)=D(z)\,\delta(z=0),\hspace{12pt}
    v_r(z)=\frac{D(z)f(z)H(z)}{f_0H_0}v_r(z=0),\hspace{12pt}
    \phi(z)=(1+z)D(z)\phi(z=0)
  \end{align}

\section{Sources}
  At each pixel, we compute the physical galaxy density for each galaxy
  type $i$ as
  \begin{equation}
    n_i(\chi,\hat{\bf n})=
    \bar{n}_i(\chi)\,
    \exp\left[D(\chi)b_i(\chi)\left(\delta_G(\chi,\hat{\bf n})-
      D(\chi)b_i(\chi)\sigma_G^2/2\right)\right],
  \end{equation}
  where $\bar{n}_i$ is the mean number density of galaxies, and $\sigma_G^2$
  is the variance of the density field at $z=0$. $\bar{n}$ is related to
  $dn/(d\Omega dz)$ as $dn/(d\Omega dz)=\bar{n}\,\chi^2/H$.

  Then, at each pixel we sample a number of galaxies from a Poisson distribution with mean
  $\lambda\equiv V_{\rm pix}n_i$, where $V_{\rm pix}$ is the comoving volume of each
  spherical voxel. We then place the resulting number of particles inside each voxel at
  random within it. Each galaxy is given a cosmological redshift corresponding $z_c$
  corresponding to its comoving distance by inverting
  \begin{equation}
    \chi=\int_0^{z_c}\frac{dz}{H(z)}.
  \end{equation}
  In addition, each galaxy is given a redshift distortion $\Delta z=v_r$ according to the
  value of the comoving velocity field in their corresponding voxel.

\section{Intensity mapping}
  The brightness temperature for a line-emitting species $a$ in a voxel $i$ is
  \begin{equation}
    T_a(\nu,\hat{\bf n})=\bar{T}_a(\nu)\left[1+\Delta^a_i(\chi\hat{\bf n})\right].
  \end{equation}
  where the mean brightness temperature is
  \begin{equation}
    \bar{T}_a(z)=\frac{3\hbar A_{21} x_2 c^2}{32\pi G k_B m_a \nu_{21}^2}
    \frac{H_0^2\Omega_{a}(z)(1+z)^2}{H(z)}
  \end{equation}
  and $\Delta^a_i$ is the redshift-space overdensity of the line-emitting species
  smoothed over the voxel. Here $x_2$ is the fraction of atoms in the excited state,
  $\Omega_a$ is the fractional density of the species, $\nu_{21}$ is the rest-frame
  frequency of the transition and $A_{21}$ is the corresponding Einstein coefficient
  for emission.

  The procedure to generate intensity maps is:
  \begin{itemize}
    \item We cycle over each voxel in the spherical shells for which we have stored
      the value of the density and velocity fields.
    \item We compute the overdensity in the voxel using a log-normal model:
      \begin{equation}
        1+\delta^a_i=
        \exp\left[D(\chi)b_a(\chi)\left(\delta_G(\chi,\hat{\bf n})-
          D(\chi)b_a(\chi)\sigma_G^2/2\right)\right].
      \end{equation}
    \item We sub-sample the voxel in $N_{\rm sub}$ random points. Each point is
      assigned a brightness temperature
      \begin{equation}
        T_{a,{\rm sub}}=\bar{T_a}(\nu)\,(1+\delta^a_i)\frac{v_{\rm vox}}{N_{\rm sub}},
      \end{equation}
      where $v_{\rm vox}$ is the comoving volume of the voxel, as well as a redshift
      displacement given by $v_r$, the value of the lightcone-evolved radial velocity
      field in the voxel.
    \item We compute the frequency channel corresponding to each point from their
      redshift (computed as the sum of its cosmological redshift and the RSD term),
      as well as the pixel index in that frequency channel corresponding to the
      angular coordinates of the point. We then add the brightness temperature of
      this point computed in the previous step to the total brightness temperature
      in the pixel.
    \item Each intensity mapping pixel is finally divided by the total comoving
      volume covered by the pixel.
  \end{itemize}
  
\section{Shear}
  We compute the shear tensor as
  \begin{align}
    &\hat{T}\equiv\left(
    \begin{array}{ccc}
      \phi_{xx}&\phi_{xy}&\phi_{xz}\\
      \phi_{yx}&\phi_{yy}&\phi_{yz}\\
      \phi_{zx}&\phi_{zy}&\phi_{zz}\\
    \end{array}\right),\hspace{12pt} 
    \hat{R}\equiv\left(
    \begin{array}{ccc}
      \sin\theta\,\cos\phi&\sin\theta\,\sin\phi&\cos\theta\\
      \cos\theta\,\cos\phi&\cos\theta\,\sin\phi&-\sin\theta\\
      -\sin\phi           &\cos\phi            &0\\
    \end{array}\right),\\
    &\hat{t}\equiv \hat{R}\cdot\hat{T}\cdot\hat{R}^T,\hspace{12pt}
    \hat{\tau}\equiv\left(
    \begin{array}{cc}
      t_{11}&t_{12}\\
      t_{21}&t_{22}
    \end{array}\right),\\
    &\hat{\Gamma}(\chi,\hat{\bf n})
    \equiv\int_0^\chi d\chi'\,\hat{\tau}(\chi'\hat{\bf n})\,
    \chi'\left(1-\frac{\chi'}{\chi}\right)=
    \int_0^\chi d\chi'\,\hat{\tau}(\chi'\hat{\bf n})\,\chi'-
    \frac{1}{\chi}\int_0^\chi d\chi'\,\hat{\tau}(\chi'\hat{\bf n})\,\chi'^2.\\
    &\hat{\Gamma}(\chi_i)=\hat{I}_{i}-\frac{1}{\chi_{i+1/2}}\hat{J}_{i},\hspace{12pt}
    \hat{I}_{i}=\hat{I}_{i-1}+\hat{\tau}(\chi_i)\frac{\chi_{i+1/2}^2-\chi_{i-1/2}^2}{2},\hspace{12pt}
    \hat{J}_{i}=\hat{J}_{i-1}+\hat{\tau}(\chi_i)\frac{\chi_{i+1/2}^3-\chi_{i-1/2}^3}{3}
  \end{align}

\section{CMB lensing}
  For $\kappa\equiv{\rm Tr}(\hat{\Gamma})/2$ we compute, from the simulation, $\kappa(\chi_{\rm max})$,
  where $\chi_{\rm max}$ is the maximum radial distance that fits in the box. In order to get CMB
  lensing we need $\kappa(\chi_{\rm LSS})=\tilde{\kappa}_{\rm max}+\Delta\kappa$, where
  \begin{align}
    \tilde{\kappa}_{\rm max}\equiv\int_0^{\chi_{\rm max}}   d\chi\delta(\chi)\frac{\chi}{2}\left(1-\frac{\chi}{\chi_{\rm LSS}}\right)\\
    \Delta\kappa\equiv\int_{\chi_{\rm max}}^{\chi_{\rm LSS}}d\chi\delta(\chi)\frac{\chi}{2}\left(1-\frac{\chi}{\chi_{\rm LSS}}\right).
  \end{align}
  The strategy to compute these is:
  \begin{itemize}
    \item We compute $\tilde{\kappa}_{\rm max}$ from $\kappa(\chi_{\rm max})$ using the same
      strategy used for shear, but taking care to divide by $\chi_{\rm LSS}$ instead of $\chi_{\rm max}$.
    \item We compute $\Delta\kappa$ as a Gaussian realization constrained to have the right correlation
      with $\tilde{\kappa}_{\rm max}$. Do do this we start by rewriting the previous equation as
      \begin{align}
        &\tilde{\kappa}_{\rm max}\equiv \int d\chi w_1(\chi)\delta(\chi\hat{\bf n}),\hspace{12pt}
        \Delta \kappa           \equiv \int d\chi w_2(\chi)\delta(\chi\hat{\bf n}),\\
        &w_1(\chi)\equiv\frac{\chi}{2}\left(1-\frac{\chi}{\chi_{\rm LSS}}\right)
        \Theta(\chi,0,\chi_{\rm max}),\hspace{12pt}
        w_2(\chi)\equiv\frac{\chi}{2}\left(1-\frac{\chi}{\chi_{\rm LSS}}\right)
        \Theta(\chi,\chi_{\rm max},\chi_{\rm LSS}).
      \end{align}
      where $\Theta(x,x_0,x_f)$ is a top-hat function.

      The covariance matrix of the two terms is therefore
      \begin{align}
        &\langle|\tilde{\kappa}_{{\rm max},\ell m}|^2\rangle=\frac{2}{\pi}\int_0^\infty dk\,k^2\,P(k)\,w^2_{1,\ell}(k),\hspace{12pt}
        \langle|\Delta\kappa_{\ell m}|^2\rangle=\frac{2}{\pi}\int_0^\infty dk\,k^2\,P(k)\,w^2_{2,\ell}(k)\\
        &\langle{\rm Re}(\tilde{\kappa}_{{\rm max},\ell m}\Delta\kappa_{\ell m})\rangle=
        \frac{2}{\pi}\int_0^\infty dk\,k^2\,P(k)\,w_{1,\ell}(k)\,w_{2,\ell}(k),\hspace{12pt}
        w_{i,\ell}(k)\equiv\int_0^\infty d\chi\,w_i(\chi)\,j_\ell(k\chi)
      \end{align}

      Thus we generate a realization of $\Delta\kappa$ at each multipole order as a Gaussian number
      with distribution $\mathcal{N}(\mu,\sigma)$, where
      \begin{align}
        \mu=\frac{\langle{\rm Re}(\tilde{\kappa}_{{\rm max},\ell m}\Delta\kappa_{\ell m})\rangle}
        {\langle|\tilde{\kappa}_{{\rm max},\ell m}|^2\rangle}\tilde{\kappa}_{{\rm max},\ell m},\hspace{12pt}
        \sigma=\langle|\Delta\kappa_{\ell m}|^2\rangle-\frac{\langle{\rm Re}(\tilde{\kappa}_{{\rm max},\ell m}\Delta\kappa_{\ell m})\rangle^2}{\langle|\tilde{\kappa}_{{\rm max},\ell m}|^2\rangle}
      \end{align}
  \end{itemize}

\section{Full-sky expressions}
The angular power spectrum between two contributions is:
\begin{equation}
 C^{ij}_\ell=4\pi\int_0^\infty \frac{dk}{k}\,\mathcal{P}_\Phi(k)\Delta^i_\ell(k)\Delta^j_\ell(k).
\end{equation}
The expressions for density, RSD, magnification, lensing convergence and CMB lensing are:
\begin{align}
  &\Delta_\ell^D(k)=\int dz p_z(z) b(z) T_\delta(k,z) j_\ell(k\chi(z))\\
  &\Delta_\ell^{RSD}(k)=\int dz \frac{(1+z) p_z(z)}{H(z)}T_\theta(k,z) j_\ell''(k\chi(z))\\
  &\Delta_\ell^M(k)=-\ell(\ell+1)\int \frac{dz}{H(z)} W^M(z) T_{\phi+\psi}(k,z) j_\ell(k\chi(z)), \\ 
  &\Delta_\ell^L(k)=-\frac{\ell(\ell+1)}{2}\int \frac{dz}{H(z)} W^L(z) T_{\phi+\psi}(k,z) j_\ell(k\chi(z)),  \\
  &\Delta_\ell^C(k)=\frac{\ell(\ell+1)}{2}\int_0^{\chi_*}d\chi
  \frac{\chi_*-\chi}{\chi\chi_*} T_{\phi+\psi}(k,z) j_\ell(k\chi),  
\end{align}
where
\begin{align}
 &W^M(z)\equiv\int_z^\infty dz' p_z(z')\frac{2-5s(z')}{2}\frac{\chi(z')-\chi(z)}{\chi(z')}\\
 &W^L(z)\equiv\int_z^\infty dz' p_z(z')\frac{\chi(z')-\chi(z)}{\chi(z')}
\end{align}

\section{Limber approximation}
The Limber approximation is
\begin{equation}
 j_\ell(x)\simeq\sqrt{\frac{\pi}{2\ell+1}}\,\delta\left(\ell+\frac{1}{2}-x\right).
\end{equation}
Thus for each $k$ and $\ell$ we can define a radial distance $\chi_\ell\equiv(\ell+1/2)/k$


\section{Expressions in the Limber approximation}
The expressions above can be written as follows in the Limber approximation. First, the power spectrum
can be rewritten as
\begin{equation}
 C^{ij}_\ell=\frac{2}{2\ell+1}\int_0^\infty dk\,P_\delta\left(k,z=0\right)
 \tilde{\Delta}^i_\ell(k)\tilde{\Delta}^j_\ell(k).
\end{equation}
where
\begin{align}
 &\tilde{\Delta}_\ell^D(k)=p_z(\chi_\ell)\,b(\chi_\ell)\,D(\chi_\ell)\,H(\chi_\ell)\\
 &\tilde{\Delta}_\ell^{RSD}(k)=
 \frac{1+8\ell}{(2\ell+1)^2}\,p_z(\chi_\ell)\,f(\chi_\ell)\,D(\chi_\ell)\,H(\chi_\ell)
 -\frac{4}{2\ell+3}\sqrt{\frac{2\ell+1}{2\ell+3}}p_z(\chi_{\ell+1})\,f(\chi_{\ell+1})\,D(\chi_{\ell+1})\,H(\chi_{\ell+1})
\longrightarrow0\\
 &\tilde{\Delta}^{ISW}_\ell(k)=
 \frac{3\Omega_{M,0}H_0^2}{k^2}H(\chi_\ell)g(\chi_\ell)\left[1-f(\chi_\ell)\right]\\
 &\tilde{\Delta}_\ell^M(k)=3\Omega_{M,0}H_0^2\frac{\ell(\ell+1)}{k^2}\,
 \frac{D(\chi_\ell)}{a(\chi_\ell)\chi_\ell}W^M(\chi_\ell)\longrightarrow
 3\Omega_{M,0}H_0^2\,\frac{\chi_\ell D(\chi_\ell)}{a(\chi_\ell)}W^M(\chi_\ell)\\
 &\tilde{\Delta}_\ell^L(k)=\frac{3}{2}\Omega_{M,0}H_0^2\sqrt{\frac{(\ell+2)!}{(\ell-2)}}\frac{1}{k^2}\,
 \frac{D(\chi_\ell)}{a(\chi_\ell)\chi_\ell}W^M(\chi_\ell)\longrightarrow
 \frac{3}{2}\Omega_{M,0}H_0^2\,\frac{\chi_\ell D(\chi_\ell)}{a(\chi_\ell)}W^M(\chi_\ell)\\
 &\tilde{\Delta}_\ell^C(k)=\frac{3}{2}\Omega_{M,0}H_0^2\frac{\ell(\ell+1)}{k^2}\,
 \frac{D(\chi_\ell)}{a(\chi_\ell)\chi_\ell}\frac{\chi_*-\chi_\ell}{\chi_*}\Theta(\chi_\ell-\chi_*)\longrightarrow
 \frac{3}{2}\Omega_{M,0}H_0^2\,\frac{\chi_\ell D(\chi_\ell)}{a(\chi_\ell)}\frac{\chi_*-\chi_\ell}{\chi_*}\Theta(\chi_\ell-\chi_*)
\end{align}



\end{document}
